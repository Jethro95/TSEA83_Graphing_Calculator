\documentclass[]{article}

%Font encoding for swedish
\usepackage[utf8]{inputenc}
\usepackage[T1]{fontenc}

\usepackage{tabularx}

%Swedish
\usepackage[swedish]{babel}

%Concentrated lists
\usepackage{mdwlist}

\usepackage{multicol}

\usepackage{tikz}
\usetikzlibrary{shapes,arrows,calc}

\usepackage[titletoc,title]{appendix}

\newcommand{\textoverline}[1]{$\overline{\mbox{#1}}$}

%opening
\title{Grafritande Räknare - Designskiss}
\author{Hannes Haglund hanha265\\Felix Härnström felha423\\Silas Lenz sille914}

\begin{document}

\maketitle
\newpage

\section{}
Vi ska implementera en generell dator av mikroprogrammerad, ej pipelinead typ. Mjukvaran skrivs i assembler. Vi har en VGA-motor, en tangentbordsavkodare och möjligtvis touchavkodare samt motor för dess skärm (utökningsmål). 

\tikzstyle{block} = [draw, fill=blue!20, rectangle, 
minimum height=3em, minimum width=6em]
\tikzstyle{sum} = [draw, fill=blue!20, circle, node distance=1cm]
\tikzstyle{input} = [coordinate]
\tikzstyle{output} = [coordinate]
\tikzstyle{pinstyle} = [pin edge={to-,thin,black}]
\begin{center}


\begin{tikzpicture}[auto, node distance=2cm,>=latex']



\node [block, node distance=3cm] (cpu) {CPU};
\node [block, below of=cpu, node distance=2cm] (memory) {Minne};
\node [output, right of=cpu] (foo) {};


\draw [<->] (cpu) -- node[near start]{} (memory);
\node [block, above of=cpu] (keyboarddriver) {Tangentbordsmotor};
\node [block, node distance=3.5cm, left of=keyboarddriver] (keyboard) {Tangentbord};

\draw [<->] (keyboarddriver) -- node {} (cpu);
\draw [->] (keyboard) -- node {} (keyboarddriver);

\node [block, node distance=2cm, below of=memory] (driver) {VGA-motor};
\node [block, right of=driver, node distance=3.5cm] (vga) {Skärm};
\draw [->] (memory) -- node {} (driver);
\draw [->] (driver) -- node {} (vga);
\draw[thick,dotted] ($(keyboarddriver.north west)+(-0.3,0.3)$)  rectangle ($(driver.south east)+(1,-0.3)$) node[pos=0] {FPGA};
\end{tikzpicture}
\end{center}

\subsection{CPU}
Vår processor är mikroprogrammerad, med delat data och programminne. Vi använder 32-bitars ordbredd. Endast blockram används.

Processor laddas alltid med samma program vid start. 

Nästan allt arbete utförs av processorn, förutom tangentbordläsning, och utritning av bildminnets innehåll. Dessa uppgifter inkluderar: beräkningar, historik, parsing av input, beräkning av graf, och så vidare.

\newpage

\subsubsection{Instruktioner}
Vi har följande adresseringsmoder:
\begin{itemize*}
\item Direkt
\item Omedelbar
\item Indirekt 
\end{itemize*}
Följande instruktionsmängd:

\begin{multicols}{5}
\begin{itemize*}
\item ADD
\item ADDF
\item SUB
\item SUBF
\item MULTF
\item DIVF
\item AND
\item ASL
\item ASR
\item BEQ
\item BMI
\item BNE
\item BRA
\item BRF
\item LOAD
\item STORE
\item HALT
\end{itemize*}
\end{multicols}

\noindent
En utförlig förklaring till varje instruktions argument, resultat, adresseringsmoder, och påverkade flaggor finns tillgängligt som bilaga, på engelska.

\subsection{Grafik} 
Vi delar upp vår display i två kolumner, där ena hälften använder tiles och andra hälften använder en bitmap i svartvitt. Räknaren (text) använder sidan med tiles, och grafen använder bitmapsidan.

Upplösning 640x480. Både tiles och bitmap i svartvitt. Uppdateringsfrekvens 60Hz.

Processorn skriver tilenummer samt bitmapen direkt till bildminnet, utan att synkronisera med bilduppritningen.

%TODO: Blockschema VGA_MOTOR

\subsection{I/O}
Input via PS/2 med en avkodare i \textsc{vhdl}. Avkodaren skriver ett tecken till en egen minnesplats som kan läsas av processorn via STORE. Vi låter instruktionen ta en virtuell adress som argument, och en viss adress som överskrider processorns minnesstorlek får referera till avkodarens minnescell.

Via en synkron \textit{read\_confirm}-signal så berättar processorn för avkodaren att den lyckats läsa ett tecken, varpå värdet på minnesplatsen nollställs och avkodaren påbörjar läsning av nästa tecken.

Hämtad input ritas ut i ett konsolfönster på skärmen, och interpreteras vid nedslag av returknappen. Tal matas in i form av flyttal (separerad med punk), och uttryck skrivs i reverse-polish-notation.

\newpage

%TODO: Blockschema tangentbordsmotor

\subsection{Minne}
Vi har följande minnen:
\begin{itemize*}
\item PC (rw)
\item ASR (rw)
\item IR (rw)
\item $\mu$PC (rw)
\item $\mu$Minne (rw)
\item Programminne (rw)
\item 6 generella dataregister (rw)
\item Statusregister (r)
\item Bildminne
\end{itemize*}
Med ordbredden 32 bitar.

\subsection{Programmering}
Vi skriver en assembler, med lite syntaktiskt socker för loopar och if-satser.

\subsection{Milstolpe}
En fungerande processor som kan rita ut flyttal från en adress i minnet med hjälp av VGA-motor.

\clearpage
\begin{appendices}
\section{Instruktionsuppsättning}
%%%%%
% ADD
%%%%%
\noindent\rule{10cm}{1pt}\newline %Horizontal line
\newcolumntype{L}{>{\raggedright\arraybackslash}X} %Custom column type for left column
\setlength\extrarowheight{5pt} %Increase row spacing
\begin{tabularx}{\textwidth}{lL}
  {\Large \textbf{ADD}} 	& {\Large \textbf{Add signed integers}}\\
  \textbf{Operation:} 		& \texttt{[destination] $\leftarrow$ [source] + [destination]}\\
  \textbf{Syntax:}  		& \texttt{ADD <ea>,Dn}\newline\texttt{ADD Dn,<ea>}\\
  \textbf{Description:}  	& Add the source operand to the destination operand and store the
result in the destination location.\\
  \textbf{Condition codes:} & \texttt{X N Z V C\newline * * * * *}\newline\newline The X-bit and C-bit are both set if carry is generated. The N-bit is set if the sum is negative. The Z-bit is set if the sum is zero. The V-bit is set if overflow occurs (in which case the Z-bit and the N-bit are undefined).\\
\end{tabularx}
\newline

%%%%%
% ADDF
%%%%%
\noindent\rule{10cm}{1pt}\newline %Horizontal line
\newcolumntype{L}{>{\raggedright\arraybackslash}X} %Custom column type for left column
\setlength\extrarowheight{5pt} %Increase row spacing
\begin{tabularx}{\textwidth}{lL}
  {\Large \textbf{ADDF}} 	& {\Large \textbf{Add signed floating-point numbers}}\\
  \textbf{Operation:} 		& \texttt{[destination] $\leftarrow$ [source] + [destination]}\\
  \textbf{Syntax:}  		& \texttt{ADDF <ea>,Dn}\newline\texttt{ADDF Dn,<ea>}\\
  \textbf{Description:}  	& Add the source operand to the destination operand and store the
result in the destination location, interpreting operands and sum as signed floating-point numbers.\\
  \textbf{Condition codes:} & \texttt{X N Z V C\newline * * * * *}\newline\newline The X-bit and C-bit are both set if carry is generated. The N-bit is set if the sum is negative. The Z-bit is set if the sum is zero. The V-bit is set if overflow occurs (in which case the Z-bit and the N-bit are undefined).\\
\end{tabularx}
\newline

\newpage

%%%%%
% SUB
%%%%%
\noindent\rule{10cm}{1pt}\newline %Horizontal line
\newcolumntype{L}{>{\raggedright\arraybackslash}X} %Custom column type for left column
\setlength\extrarowheight{5pt} %Increase row spacing
\begin{tabularx}{\textwidth}{lL}
  {\Large \textbf{SUB}} 	& {\Large \textbf{Subtract signed integers}}\\
  \textbf{Operation:} 		& \texttt{[destination] $\leftarrow$ [source] - [destination]}\\
  \textbf{Syntax:}  		& \texttt{SUB <ea>,Dn}\newline\texttt{SUB Dn,<ea>}\\
  \textbf{Description:}  	& Subtract the destination operand from the source operand and store the
result in the destination location.\\
  \textbf{Condition codes:} & \texttt{X N Z V C\newline * * * * *}\newline\newline The X-bit and C-bit are both set if carry is generated. The N-bit is set if the difference is negative. The Z-bit is set if the difference is zero. The V-bit is set if overflow occurs (in which case the Z-bit and the N-bit are undefined).\\
\end{tabularx}
\newline

%%%%%
% SUBF
%%%%%
\noindent\rule{10cm}{1pt}\newline %Horizontal line
\newcolumntype{L}{>{\raggedright\arraybackslash}X} %Custom column type for left column
\setlength\extrarowheight{5pt} %Increase row spacing
\begin{tabularx}{\textwidth}{lL}
  {\Large \textbf{SUBF}} 	& {\Large \textbf{Subtract signed floating-point numbers}}\\
  \textbf{Operation:} 		& \texttt{[destination] $\leftarrow$ [source] - [destination]}\\
  \textbf{Syntax:}  		& \texttt{SUBF <ea>,Dn}\newline\texttt{SUBF Dn,<ea>}\\
  \textbf{Description:}  	& Subtract the destination operand from the source operand and store the
result in the destination location, interpreting operands and difference as signed floating-point numbers.\\
  \textbf{Condition codes:} & \texttt{X N Z V C\newline * * * * *}\newline\newline The X-bit and C-bit are both set if carry is generated. The N-bit is set if the difference is negative. The Z-bit is set if the difference is zero. The V-bit is set if overflow occurs (in which case the Z-bit and the N-bit are undefined).\\
\end{tabularx}
\newline

\newpage

%%%%%
% DIVF
%%%%%
\noindent\rule{10cm}{1pt}\newline %Horizontal line
\newcolumntype{L}{>{\raggedright\arraybackslash}X} %Custom column type for left column
\setlength\extrarowheight{5pt} %Increase row spacing
\begin{tabularx}{\textwidth}{lL}
  {\Large \textbf{DIVF}} 	& {\Large \textbf{Signed floating-point divide}}\\
  \textbf{Operation:} 		& \texttt{[destination] $\leftarrow$ [destination]/[source]}\\
  \textbf{Syntax:}  		& \texttt{DIVF <ea>,Dn}\\
  \textbf{Description:}  	& Divide the destination operand by the source operand and store
the result in the destination, interpreting operands and result as signed floating-point numbers.\\
  \textbf{Condition codes:} & \texttt{X N Z V C\newline - * * * 0}\newline\newline The X-bit is not affected by a division. The N-bit is set if the
quotient is negative. The Z-bit is set if the quotient is zero. The Vbit
is set if division overflow occurs (in which case the Z- and Nbits
are undefined). The C-bit is always cleared.\\
\end{tabularx}
\newline

%%%%%
% MULTF
%%%%%
\noindent\rule{10cm}{1pt}\newline %Horizontal line
\newcolumntype{L}{>{\raggedright\arraybackslash}X} %Custom column type for left column
\setlength\extrarowheight{5pt} %Increase row spacing
\begin{tabularx}{\textwidth}{lL}
  {\Large \textbf{MULTF}} 	& {\Large \textbf{Signed floating-point multiply}}\\
  \textbf{Operation:} 		& \texttt{[destination] $\leftarrow$ [destination]$\times$[source]}\\
  \textbf{Syntax:}  		& \texttt{MULTF <ea>,Dn}\\
  \textbf{Description:}  	& Multiply the destination operand by the source operand and store
the result in the destination, interpreting operands and result as signed floating-point numbers.\\
  \textbf{Condition codes:} & \texttt{X N Z V C\newline - * * * 0}\newline\newline The X-bit is not affected by a multiplication. The N-bit is set if the 
product is negative. The Z-bit is set if the product is zero. The V-bit
is set if division overflow occurs (in which case the Z-bit and the N-bit
are undefined). The C-bit is always cleared.\\
\end{tabularx}
\newline

\newpage

%%%%%
% AND
%%%%%
\noindent\rule{10cm}{1pt}\newline %Horizontal line
\newcolumntype{L}{>{\raggedright\arraybackslash}X} %Custom column type for left column
\setlength\extrarowheight{5pt} %Increase row spacing
\begin{tabularx}{\textwidth}{lL}
  {\Large \textbf{AND}} 	& {\Large \textbf{AND logical}}\\
  \textbf{Operation:} 		& \texttt{[destination] $\leftarrow$ [source].[destination]}\\
  \textbf{Syntax:}  		& \texttt{AND <ea>,Dn}\newline\texttt{AND Dn,<ea>}\\
  \textbf{Description:}  	& AND the source operand to the destination operand and store the
result in the destination location.\\
  \textbf{Condition codes:} & \texttt{X N Z V C\newline - * * 0 0}\newline\newline The N-bit is set to the most significant bit of the result. The Z-bit is set if the result is equal to zero.\\
\end{tabularx}
\newline

\newpage

%%%%%
% ASL/ASR
%%%%%
\noindent\rule{10cm}{1pt}\newline %Horizontal line
\newcolumntype{L}{>{\raggedright\arraybackslash}X} %Custom column type for left column
\setlength\extrarowheight{5pt} %Increase row spacing
\begin{tabularx}{\textwidth}{lL}
  {\Large \textbf{ASR}} 	& {\Large \textbf{Arithmetic shift left/right}}\\
  \textbf{Operation:} 		& \texttt{[destination] $\leftarrow$ [destination] shifted by <count>}\\
  \textbf{Syntax:}  		& \texttt{ASL <ea>,Dn}\newline
  							  \texttt{ASR <ea>,Dn}\newline
 							  \texttt{ASL \#<data>,Dy}\newline
 							  \texttt{ASR \#<data>,Dy}\newline
 							  \texttt{ASL <ea>}\newline
 							  \texttt{ASR <ea>}\newline 							  
 							  \\
  \textbf{Description:}  	& Arithmetically shift the bits of the operand in the specified direction
(i.e., left or right). The shift count may be specified in one of
three ways. The count may be a literal, the contents of a data
register, or the value 1. An immediate (i.e., literal) count permits
a shift of 1 to 8 places. If the count is in a register, the value is
modulo 64 (i.e., 0 to 63). If no count is specified, one shift is made
(i.e., ASL <ea> shifts the contents of the word at the effective
address one place left).

The effect of an arithmetic shift left is to shift a zero into the
least-significant bit position and to shift the most-significant bit
out into both the X- and the C-bits of the CCR. The overflow bit
of the CCR is set if a sign change occurs during shifting (i.e., if
the most-significant bit changes value during shifting).

The effect of an arithmetic shift right is to shift the leastsignificant
bit into both the X- and C-bits of the CCR. The mostsignificant
bit (i.e., the sign bit) is replicated to preserve the sign of
the number.\\
  \textbf{Condition codes:} & \texttt{X N Z V C\newline * * * * *}\\
\end{tabularx}
\newline

\newpage

%%%%%
% Bcc
%%%%%
\noindent\rule{10cm}{1pt}\newline %Horizontal line
\newcolumntype{L}{>{\raggedright\arraybackslash}X} %Custom column type for left column
\setlength\extrarowheight{5pt} %Increase row spacing
\begin{tabularx}{\textwidth}{lL}
  {\Large \textbf{Bcc}} 	& {\Large \textbf{Branch on condition cc}}\\
  \textbf{Operation:} 		& \texttt{If cc = 1 THEN [PC] $\leftarrow$ [PC] + d}\\
  \textbf{Syntax:}  		& \texttt{Bcc <label>}\\
  \textbf{Description:}  	& If the specified logical condition is met, program execution
continues at location [PC] + displacement, d.\newline\newline
							  \texttt{BEQ} {} {} {} branch on equal\hfill \texttt{Z}\newline
							  \texttt{BMI} {} {} {} branch on minus\hfill \texttt{N}\newline
							  \texttt{BMI} {} {} {} branch on not equal\hfill \texttt{\textoverline{Z}}\newline
							  \texttt{BRF} {} {} {} branch on overflow set\hfill \texttt{V}\newline
							  \\
  \textbf{Condition codes:} & \texttt{X N Z V C\newline - - - - -}\\
\end{tabularx}
\newline

%%%%%
% BRA
%%%%%
\noindent\rule{10cm}{1pt}\newline %Horizontal line
\newcolumntype{L}{>{\raggedright\arraybackslash}X} %Custom column type for left column
\setlength\extrarowheight{5pt} %Increase row spacing
\begin{tabularx}{\textwidth}{lL}
  {\Large \textbf{BRA}} 	& {\Large \textbf{Branch always}}\\
  \textbf{Operation:} 		& \texttt{[PC] $\leftarrow$ [PC] + d}\\
  \textbf{Syntax:}  		& \texttt{BRA <label>}\newline\texttt{BRA <literal>}\\
  \textbf{Description:}  	& Program execution continues at location [PC] + d.\\
  \textbf{Condition codes:} & \texttt{X N Z V C\newline - - - - -}\\
\end{tabularx}
\newline

\newpage

%%%%%
% LOAD
%%%%%
\noindent\rule{10cm}{1pt}\newline %Horizontal line
\newcolumntype{L}{>{\raggedright\arraybackslash}X} %Custom column type for left column
\setlength\extrarowheight{5pt} %Increase row spacing
\begin{tabularx}{\textwidth}{lL}
  {\Large \textbf{LOAD}} 	& {\Large \textbf{Load value}}\\
  \textbf{Operation:} 		& \texttt{[data register] $\leftarrow$ <data>}\\
  \textbf{Syntax:}  		& \texttt{LOAD <ea>,Dn}\newline\texttt{LOAD \#<data>,Dn}\\
  \textbf{Description:}  	& Write to data register, where the data depends on the adressing mode. With direct adressing, it is the memory contents at the given address. With immediate, the given literal.\\
  \textbf{Condition codes:} & \texttt{X N Z V C\newline - - - - -}\\
\end{tabularx}
\newline

%%%%%
% STORE
%%%%%
\noindent\rule{10cm}{1pt}\newline %Horizontal line
\newcolumntype{L}{>{\raggedright\arraybackslash}X} %Custom column type for left column
\setlength\extrarowheight{5pt} %Increase row spacing
\begin{tabularx}{\textwidth}{lL}
  {\Large \textbf{STORE}} 	& {\Large \textbf{Store value}}\\
  \textbf{Operation:} 		& \texttt{[memory] $\leftarrow$ <data>}\\
  \textbf{Syntax:}  		& \texttt{STORE Dn,<ea>}\newline\texttt{STORE \#<data>,<ea>}\\
  \textbf{Description:}  	& Write to main memory. If a data register is given, its contents are written. If immediate addressing is used, a given literal is written.\\
  \textbf{Condition codes:} & \texttt{X N Z V C\newline - - - - -}\\
\end{tabularx}
\newline

%%%%%
% HALT
%%%%%
\noindent\rule{10cm}{1pt}\newline %Horizontal line
\newcolumntype{L}{>{\raggedright\arraybackslash}X} %Custom column type for left column
\setlength\extrarowheight{5pt} %Increase row spacing
\begin{tabularx}{\textwidth}{lL}
  {\Large \textbf{HALT}} 	& {\Large \textbf{Halt execution}}\\
  \textbf{Operation:} 		& \texttt{HALT}\\
  \textbf{Syntax:}  		& \texttt{HALT}\\
  \textbf{Description:}  	& Processor suspends all processing.\\
  \textbf{Condition codes:} & \texttt{X N Z V C\newline 0 0 0 0 0}\\
\end{tabularx}
\newline

\end{appendices}
\end{document}
