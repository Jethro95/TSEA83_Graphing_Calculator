\documentclass[]{article}
%Font encoding for swedish
\usepackage[utf8]{inputenc}
\usepackage[T1]{fontenc}


% Title Page
\title{Teknisk rapport\\Grafisk räknare}
\author{}


\begin{document}
\maketitle

\begin{abstract}
\end{abstract}
\section{Inledning}

\section{Apparaten}
Börja gärna med en bild på bygget och redgör för hur apparaten används. Detta blir en kombinerad presentation av konstruktionen och användarhandledning. 
\section{Teori}
Här kan ett videoprojekt beskriva videoformatet, ett MIDI-projekt redogöra för MIDI-standarden ...
\subsection{Reverse polish}
\section{Hårdvaran}
Börja med ett översiktligt blockschema, gå vidare till mera detaljerade blockschemor. Vill man rita figurer och blockschemor på datorn så är Inkscape ett bra alternativ. Handritade figurer och blockschemor är acceptabelt, bara de är snygga och väl läsbara, MEN dessa ska då vara inscannade, dvs avfotograferade figurer och blockschemor är INTE godtagbart.
Beskriv sedan hur de olika blocken fungerar. Tänk på läsbarheten och växla mellan figurer och text. Den här typen av text är ganska "grafisk", då nästan varje mening syftar in i en figur. Var därför noga med att text och figurer stämmer överens.
\section{Slutsatser}
\section{Referenser}
\section{VHDL-dokumentation}
Inte själva VHDL-koden utan en förklaring av den. Typiskt en förklaring av varje process.
\end{document}          
